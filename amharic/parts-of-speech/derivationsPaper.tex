
\documentclass[a4paper]{report}
\usepackage{fancyhdr}
\pagestyle{fancy}
\lhead{\textit{dmulholl@cs.indiana.edu}}
%\chead{\emph{Http://Tobia.EthiopiaOnline.Net}}
\rhead{\textit{\today}}

\newcommand{\tableTitleA}[1]{Allowable Prefixes & \dotable{Required}{Midfix} & #1 Stem & Allowable Suffixes  \\ \hline}
\newcommand{\tableTitleB}[1]{Allowable Prefixes & \dotable{Required}{Midfix} & \dotable{#1}{Stem} & Allowable Suffixes  \\ \hline}
\newcommand{\tableTitleC}[1]{\dotableL{Allowable}{Prefixes} & \dotable{Required}{Midfix} & \dotable{#1}{Stem} & \dotableR{Allowable}{Suffixes}  \\ \hline}
\newcommand{\myraise}[1]{\raisebox{0.06ex}{#1}}

\def\lbR{[{\lG},{\bG}]}
\def\lb{(([{\lG},{\bG}]+OP$_{\tinyl,\tinyb}$)}
\def\lbu{(([{\lG},{\bG}]+OP$_{\tinyl,\tinyb}$),DEF)}
\def\lbuR{[([{\lG},{\bG}]+OP$_{\tinyl,\tinyb}$),DEF]}

\def\lbbet{([{\lG},{\bG}]+OP$_{\tinyl,\tinyb}$)}
\def\lbbetu{(([{\lG},{\bG}]+OP$_{\tinyl,\tinyb}$),DEF)}
\def\lbbetuR{[([{\lG},{\bG}]+OP$_{\tinyl,\tinyb}$),DEF]}
\def\continuants{((({\mG},{\sG}) + {\naG}),{\maG})}
\def\continuantssa{((({\mG},{\sG}) + {\naG}),{\maG},{\saG})}
\def\continuantsA{(({\mG},{\sG}) + ({\naG}+{\AG}))}
\def\continuantsx{({\mG},{\sG},{\maG},{\naG})}
\def\continuantsFour{({\mG},{\sG},{\maG},{\saG})}
\def\continuantsxsa{({\mG},{\sG},{\maG},{\saG},{\naG})}
\def\continuantsxa{({\mG},{\sG},{\naG},{\maG},{\AG})}
\def\continuantsn{({\mG},{\sG},{\nG},{\maG},{\naG})}
\def\continuantsy{(({\mG}+{\naG}),{\sG},{\maG})}
\def\continuantsz{({\mG},{\sG}) + ({\naG},{\maG},{\saG})}
\def\continuantsga{(({\mG},{\sG})+{\gaG}+{\naG}),({\gaG}+\continuants)}
\def\continuantssaga{(({\mG},{\sG})+{\gaG}+{\naG}),({\gaG}+\continuantssa)}
\def\continuantsgaz{({\mG},{\sG}) + {\gaG} + \continuantsFour}
\def\continuantsgazna{({\mG},{\sG}) + {\gaG} + \continuantssa$^{\tinyga}$}
\def\continuantsgazz{({\mG},{\sG}) + {\gaG} + ({\mG},{\sG},{\maG})$^{\tinyga}$}
%  under the last sequence the combinations of
%  ({\mG},{\sG}) + {\gaG} + ({\mG},{\sG})...{\maG},{\saG} follow as:
%  {\mG}{\gaG}{\mG}, {\mG}{\gaG}{\sG}, and {\sG}{\gaG}{\sG}
%  {\sG}{\gaG}{\mG} does NOT form nor ({\maG},{\saG}) can not follow ({\mG},{\sG}) + {\gaG}  
%
%
\def\tinyl{\raisebox{-0.6ex}{\tinyet{\char204}}}
\def\tinysa{\raisebox{-0.6ex}{\tinyet{\char148}}}
\def\tinyn{\raisebox{-0.6ex}{\tinyet{\char216}}}
\def\tinyna{\raisebox{-0.6ex}{\tinyet{\char214}}}
\def\tinyw{\raisebox{-0.6ex}{\tinyet{\char074}}}
\def\tinyb{\raisebox{-0.6ex}{\tinyet{\char144}}}
\def\tinyIye{\raisebox{-0.6ex}{\tinyet{\char138}{\char105}}}
\def\tinynet{\raisebox{-0.6ex}{\tinyet{\char110}{\char252}}}
\def\tinyit{\raisebox{-0.6ex}{\tinyet{\char135\char252}}}
\def\tinyInd{\raisebox{-0.6ex}{\tinyet{\char138}{\char216}{\char156}}}
\def\tinyIsk{\raisebox{-0.6ex}{\tinyet{\char138}{\char150}{\char234}}}
\def\tinyale{\raisebox{-0.6ex}{\tinyet{\char97}{\char108}}}
\def\tinyIne{\raisebox{-0.6ex}{\tinyet{\char138}{\char110}}}
\def\smallne{\raisebox{-0.6ex}{\smallet{\char110}}}
\def\tinysEt{\raisebox{-0.6ex}{\tinyet{\char149\char252}}}
\def\tinynu{\raisebox{-0.6ex}{\tinyet{\char212}}}
\def\tinyNa{\raisebox{-0.6ex}{\tinyet{\char220}}}
\def\tinyNaw{\raisebox{-0.6ex}{\tinyet{\char220}{\char074}}}
\def\tinyga{\raisebox{-0.6ex}{\tinyet{\char172}}}
\def\downstar{\raisebox{-0.6ex}{\tiny{$\star$}}}
\def\downstarnet{\raisebox{-0.6ex}{\tiny{$\star$},\tinynet}}
\def\upstar{$^\star$}

\def\noi{\noindent}
\newcommand{\dotable}[2]{\begin{tabular}{c} #1 \\ #2 \end{tabular}}
\newcommand{\dotableL}[2]{\begin{tabular}{l} #1 \\ #2 \end{tabular}}
\newcommand{\dotableR}[2]{\begin{tabular}{r} #1 \\ #2 \end{tabular}}
%\newcommand{\shadecell}[1]{\multicolumn{1}{>{\columncolor[gray]{0.7}}c}{#1}}
%\newcommand{\shadecell}[1]{\begin{tabular}{>{\columncolor[gray]{0.7}}c} #1 \end{tabular}}
\newcommand{\shadecell}[1]{\psboxit{box 0.7 setgray fill}{\rule[-5mm]{0mm}{14mm}\makebox[15mm]{#1}}} 
\newcommand{\shadehalfcell}[1]{\psboxit{box 0.7 setgray fill}{\rule[-2mm]{0mm}{7mm}\makebox[14mm]{#1}}} 

\newcommand{\xa}[6]{\begin{tabular}{*{6}{@{}c@{}|}}
    {\,~~\,}
 &  \multicolumn{2}{@{}c@{}|}{$\;${\smallet \,\char97\,}$\;$}    % a
 &  {\smallet \char97\char150}                                   % as
 &  \multicolumn{2}{@{}c@{}|}{$\;${\smallet \,\char116\,}$\;$}   % te
\\  \hline
          #1  &  #2  &  #3  &  #4  &  #5  &  #6     \\ \hline
 \end{tabular}}


\newcommand{\xb}[6]{\begin{tabular}{*{6}{@{}c@{}|}}
     \multicolumn{2}{@{}c@{}|}{\smallet \char97\char150\char116}  % aste / teste
  &  \multicolumn{2}{@{}c@{}|}{\smallet \,\char97\char216\,}      % an   / ten
  &  \multicolumn{2}{@{}c@{}|}{\smallet \,\char97\char117\,}      % ax   / tex
  \\  \hline
          #1  &  #2  &  #3  &  #4  &  #5  &  #6     \\ \hline
 \end{tabular}}

\newcommand{\xc}[6]{\begin{tabular}{*{5}{@{}c@{}|}@{}c@{}}
          #1  &  #2  &  #3  &  #4  &  #5  &  #6     \\ \hline
 \end{tabular}}




%\def\xme{\begin{tabular}{*{5}{@{}c@{}|}*5{c|}@{}c@{}}
\def\xme{\begin{tabular}{*{11}{@{}c@{}|}@{}c@{}}
%\def\xme{\begin{tabular}{*{8}{@{\,}c@{\,}|}@{\,}c@{\,}}
     {~~}
  &  {\smallet \hspace*{1pt}\char97\hspace*{1pt}}                    % a
  &  {\smallet \char97\char150}                                      % as
  &  {\smallet \char97\char116$\!$}                                  % at
  &  \multicolumn{2}{@{}c@{}|}{\smallet {$\;\,$\char116$\;\,$}}                  % te
  &  \multicolumn{2}{@{}c@{}|}{\smallet \char97\char150\char116}     % aste / teste
  &  \multicolumn{2}{@{}c@{}|}{\smallet \,\char97\char216\,}         % an   / ten
  &  \multicolumn{2}{@{}c@{}}{\smallet \char97\char117}              % ax   / tex
  \\  \hline
% |  a  |  as  | at |  te1  | te2 | aste1| aste2|  an1 | an2| ax2 | ax2
  &     &      &    &~$\;\,$&     & ~~\, &      &  ~~  &    & ~\, &    \\ \hline
  &     &      &    &~$\;\,$&     & ~~\, &      &  ~~  &    & ~\, &    \\ 
\end{tabular}}

\def\myhead{
     {\,~~\,}
  &  {\smallet \,\char97\,}                                          % a
  &  {\smallet \char97\char150}                                      % as
  &  {\smallet \,\char116\,}                                         % te
  &  \multicolumn{2}{@{}c@{}|}{\smallet \,\char97\char116\,}         % at 
  &  \multicolumn{2}{@{}c@{}|}{\smallet \char97\char150\char116}     % aste / teste
  &  \multicolumn{2}{@{}c@{}|}{\smallet \char97\char216}         % an   / ten
  &  \multicolumn{2}{@{}c@{}}{\smallet \char97\char117}             % ax   / tex
  \\  \hline
}

% \def\xme{\begin{tabular}{*{11}{@{}c@{}|}}
% \def\xme{\begin{tabular}{*{10}{@{}c@{}|}@{}c@{}}
%   \multicolumn{2}{@{}c@{}|}{}                            &
%   \multicolumn{3}{@{}c@{}|}{{\smallet \char97}}          &
%   \multicolumn{3}{@{}c@{}|}{{\smallet \char97\char150}}  &
%   \multicolumn{3}{@{}c@{}|}{{\smallet \char116}}         \\ \hline
%   \scriptsize{\,}\small{S}\scriptsize{\,} & \scriptsize{\,}\scriptsize{F}\scriptsize{\,} &
%   \scriptsize{\,}\small{S}\scriptsize{\,} & \scriptsize{\,}\small{C}\scriptsize{\,} & \scriptsize{\,}\small{F}\scriptsize{\,} &
%   \scriptsize{\,}\small{S}\scriptsize{\,} & \scriptsize{\,}\small{C}\scriptsize{\,} & \scriptsize{\,}\small{F}\scriptsize{\,} &
%   \scriptsize{\,}\small{S}\scriptsize{\,} & \scriptsize{\,}\small{C}\scriptsize{\,} & \scriptsize{\,}\small{F}\scriptsize{\,} \\ \hline
%                 &              &
%                 &              &              &
%                 &              &              &
%                 &              & 
% \end{tabular}}

\input{ethnfss}
\newfont{\smallet}{ethio8}
\newfont{\tinyet}{ethio6}
\usepackage{array}

\begin{document}

\subsection*{Document Status, Notices, Introduction}

The status of this document is \textit{draft}.  The information presented remains under a development status and is expected to change as the document advances.  The material herein is presented to be insightful and not factual at this time.  However, though the accuracy of the verb and noun forms presented will improve, the estimated number of conjugations will likely not be effected so as to impair the purpose of the presentation -which is to present a minimal number of word forms possible from a typical Amharic stem.  Indeed, further study has to date only turned up greater and not fewer numbers of derived word forms.\\

This document may be redistributed to all who find it useful provided that this notice remains.  Exerts from this document used elsewhere should give the proper reference to this
document and its author.\\



\subsection*{Stems}

Amharic verb stems are observed to follow an underlying classic Semitic
pattern of \texttt{CCC}.  As the author is not an expert on Semitic languages
further analysis in this vein is avoided.  Amharic verbs may have from generally 2-5 radicals, though verbs with 4 and 5 radicals are fewer and often can be shown to be forms of 2 and 3 radical verbs (owing to reduplicative phenomena). A single verb, {\xaG}, exists having only one radical though is the subject of some disagreement.\\

In their skeletal form (as a \texttt{CCC} sequence) verbs are without meaning in Amharic, meaning is not realized until the stem is impregnated by at least one vowel.  Orthographically a \texttt{Cv} sequence emerges as a change in an Amharic radical's syllabic order.  The Amharic syllabary allows for up to 8 syllabic orders for most consonant series, 4 consonants may have up to 12 forms.  The 6$^{th}$ order is generally equated to an alphabetic consonant (the vowel component may or may not be voiced). Forms after the 7$^{th}$ are diphthongs.  Before inflection for person a stem may not be impregnated with all of the 7 available vowels.  In 3, 4 and 5 radical verbs only \texttt{\"{a}} and \texttt{a} are possible.  In 2 radical verbs \texttt{e} and \texttt{o} are additional possibilities.  The Amharic {\eG} (\texttt{a}) is the only vowel that may stand alone as a radical.  It appears at the start of a 3 radical verb and will later have some consequence on elisions with prefixes.\\

Verbs stems in Amharic stand at a higher potential state than those of nouns.  That is noun stems will be realized into fewer surface level forms as the nature of noun morphology ties nouns first to relatively fewer available contextually based classifications.  Verbs morphology is contextually sensitive as well however any verb stem will have a high number of tense based derivatives in which it may be realized.  Additionally, verbs may take on noun forms thru noun of agent, infinitive and instrumental forms.  A family of words ({\sseG}{\raG}, {\fG}{\qG}{\rG}, {\leG}{\bG}{\sG}, etc) have stems that can equally be realized as nouns or verbs. \\

Most work on Amharic (such as Cohen) presents 7 tenses and 6 ``themes'' for verb stems before personal inflections.  Wolf Leslau demonstrates an Amharic extremum of 12 possible preinflected states.  Verbs that may be realized in the additional 5 states are rare.  No verb will be realized in all states, the states that a verb can hold is dependent on context.  A table of the Amharic potential extremum is presented for the classic Type 3-A verb ``{\neG}{\geG}{\rG}'':


\noi
\subsection*{Conjugations of {\neG}{\geG}{\reG}}
\hspace*{-0.50in}
\begin{tabular}{|*{8}{c|}} \hline
              &  Simple  &    {\teG}                        &  {\eG}                                & {\eG}{\sG}
              & {\eG}{\sG}{\teG}                                  &  {\eG}{\nG}                              & {\eG}{\xG} \\ \hline
  Perfect     &   {\neG}\geminateG{\geG}{\reG} & \dotable{{\teG}{\neG}\geminateG{\geG}{\reG}}{{\teG}{\naG}\geminateG{\geG}{\reG}} & \dotable{({\eG}{\neG}\geminateG{\geG}{\reG})}{{\eG}\geminateG{\naG}\geminateG{\geG}{\reG}}     & {\eG}{\sG}{\neG}\geminateG{\geG}{\reG} 
              & \dotable{({\eG}{\sG}{\teG}{\naG}\geminateG{\geG}{\reG})}{({\teG}{\sG}{\teG}{\naG}\geminateG{\geG}{\reG})}& \dotable{({\eG}{\naG}{\neG}\geminateG{\geG}{\reG})}{({\teG}{\nG}{\naG}\geminateG{\geG}{\reG})}     & \dotable{({\eG}{\xG}{\neG}\geminateG{\geG}{\reG})}{({\teG}{\xG}{\neG}\geminateG{\geG}{\reG})} \\ \hline 
  Imperfect   & {\yG}{\neG}{\gG}{\raG}{\lG} & \dotable{{\yG}\geminateG{\neG}\geminateG{\geG}{\raG}{\lG}}{{\yG}\geminateG{\naG}\geminateG{\geG}{\raG}{\lG}} & \dotable{({\yaG}{\neG}{\gG}{\raG}{\lG})}{{\yaG}\geminateG{\naG}\geminateG{\gG}{\raG}{\lG}} & {\yaG}{\sG}{\neG}\geminateG{\gG}{\raG}{\lG} 
              & \dotable{({\yaG}{\sG}{\teG}{\naG}\geminateG{\gG}{\raG}{\lG})}{({\yG}{\teG}{\sG}{\teG}{\naG}\geminateG{\gG}{\raG}{\lG})}& \dotable{({\yaG}{\nG}{\neG}\geminateG{\gG}{\raG}{\lG})}{({\yG}{\nG}{\neG}\geminateG{\geG}{\raG}{\lG})} & \dotable{({\eG}{\xG}{\neG}\geminateG{\geG}{\raG}{\lG})}{({\teG}{\xG}{\neG}\geminateG{\geG}{\raG}{\lG})} \\ \hline 
  Jussive     & {\yG}{\nG}{\geG}{\rG} &  \dotable{{\yG}\geminateG{\neG}{\geG}{\rG}}{{\yG}\geminateG{\naG}{\geG}{\rG}}& \dotable{({\yaG}{\nG}{\gG}{\rG})}{{\yaG}\geminateG{\naG}{\gG}{\rG}}     & {\yaG}{\sG}{\neG}\geminateG{\gG}{\rG} 
              & \dotable{({\yaG}{\sG}{\teG}{\naG}{\gG}{\rG})}{({\yG}{\teG}{\sG}{\teG}{\naG}{\gG}{\rG})}  & \dotable{({\yaG}{\nG}{\naG}\geminateG{\gG})}{({\yG}{\nG}{\naG}{\geG}{\rG})}   & \dotable{({\eG}{\xG}{\neG}\geminateG{\geG}{\reG})}{({\teG}{\xG}{\neG}\geminateG{\geG}{\reG})} \\ \hline 
  Gerund      &   {\neG}{\gG}{\roG} &  \dotable{{\teG}{\neG}{\gG}{\roG}}{{\teG}{\naG}{\gG}{\roG}}& \dotable{({\eG}{\nG}{\gG}{\roG})}{{\eG}\geminateG{\naG}{\gG}{\roG}}     & {\eG}{\sG}{\neG}\geminateG{\gG}{\roG} 
              & \dotable{({\eG}{\sG}{\teG}{\naG}{\gG}{\roG})}{({\teG}{\sG}{\teG}{\naG}{\gG}{\roG})}    & \dotable{({\eG}{\nG}{\naG}\geminateG{\goG})}{({\teG}{\nG}{\naG}\geminateG{\goG})}     & \dotable{({\eG}{\xG}{\neG}\geminateG{\geG}{\reG})}{({\teG}{\xG}{\neG}\geminateG{\geG}{\reG})} \\ \hline 
  NOA         &   {\neG}{\gaG}{\riG} &  \dotable{{\teG}{\neG}{\gaG}{\riG}}{{\teG}{\naG}{\gaG}{\riG}}& \dotable{({\eG}{\neG}{\gaG}{\riG})}{{\eG}\geminateG{\naG}{\gaG}{\riG}}     & {\eG}{\sG}{\neG}\geminateG{\gaG}{\riG}  
              & \dotable{({\eG}{\sG}{\teG}{\naG}{\gaG}{\riG})}{({\teG}{\sG}{\teG}{\naG}{\gaG}{\riG})}    & \dotable{({\eG}{\nG}{\naG}{\gaG}{\riG})}{({\teG}{\nG}{\naG}{\gaG}{\riG})} & \dotable{({\eG}{\xG}{\neG}\geminateG{\geG}{\reG})}{({\teG}{\xG}{\neG}\geminateG{\geG}{\reG})} \\ \hline 
  Infinitive  & {\meG}{\nG}{\geG}{\rG} &  \dotable{{\meG}\geminateG{\neG}{\geG}{\rG}}{{\meG}\geminateG{\naG}{\geG}{\rG}}& \dotable{({\maG}{\nG}{\geG}{\rG})}{{\maG}\geminateG{\naG}{\geG}{\rG}}     & {\maG}{\sG}{\neG}\geminateG{\geG}{\rG}  
              & \dotable{({\maG}{\sG}{\teG}{\naG}{\geG}{\rG})}{({\meG}{\teG}{\sG}{\teG}{\naG}{\geG}{\rG})}  & \dotable{({\eG}{\nG}{\naG}{\gaG}{\riG})}{({\teG}{\nG}{\naG}{\gaG}{\riG})} & \dotable{({\eG}{\xG}{\neG}\geminateG{\geG}{\reG})}{({\teG}{\xG}{\neG}\geminateG{\geG}{\reG})} \\ \hline 
  Instrumental&{\meG}{\nG}{\geG}{\riG}{\yaG}& \dotable{{\meG}\geminateG{\neG}{\geG}{\riG}{\yaG}}{{\meG}\geminateG{\naG}{\geG}{\riG}{\yaG}} & \dotable{({\maG}{\nG}{\geG}{\riG}{\yaG})}{{\maG}\geminateG{\naG}{\geG}{\riG}{\yaG}}     & {\maG}{\sG}{\neG}\geminateG{\geG}{\riG}{\yaG}  
              & \dotable{({\maG}{\sG}{\teG}{\naG}{\geG}{\riG}{\yaG})}{({\meG}{\teG}{\sG}{\teG}{\naG}{\geG}{\riG}{\yaG})} & \dotable{({\maG}{\nG}{\naG}{\geG}{\riG}{\yaG})}{({\meG}{\nG}{\naG}{\geG}{\riG}{\yaG})} & \dotable{({\eG}{\xG}{\neG}\geminateG{\geG}{\reG})}{({\teG}{\xG}{\neG}\geminateG{\geG}{\reG})} \\ \hline 

\end{tabular}\\



The imperative form of a verb is usually treated as the 2$^{nd}$ person singular form of the jussive.  This is common because the 2$^{nd}$ person singular in the jussive is usually, \emph{but not always}, unused and the imperative has no form outside of the 2$^{nd}$ person singular.  Care is needed here.  The relative form of verbs, not shown, are born from the contingent.  The contingent is implied in the above table through the imperfect and jussive.\\

The Amharic verb potential space is not complete without considering also the frequentative themes:

\noi
\subsection*{Conjugations of {\neG}{\gaG}{\geG}{\reG}}
\hspace*{-0.70in}
\begin{tabular}{|*{8}{c|}} \hline
              &  Simple    &    {\teG}                            &  {\eG}                                  & {\eG}{\sG}
              & {\eG}{\sG}{\teG}                                        &  {\eG}{\nG}                                & {\eG}{\xG} \\ \hline
  Perfect     &   {\neG}{\gaG}\geminateG{\geG}{\reG} & \dotable{{\teG}{\neG}{\gaG}\geminateG{\geG}{\reG}}{({\teG}\geminateG{\neG}{\gaG}\geminateG{\geG}{\reG})} & \dotable{{\eG}{\neG}{\gaG}\geminateG{\geG}{\reG}}{{\eG}\geminateG{\neG}{\gaG}\geminateG{\geG}{\reG}}     & {\eG}{\sG}{\neG}{\gaG}\geminateG{\geG}{\reG} 
              & \dotable{({\eG}{\sG}{\teG}{\naG}{\gaG}\geminateG{\geG}{\reG})}{({\teG}{\sG}{\teG}{\naG}\geminateG{\geG}{\reG})}    & \dotable{{\eG}{\naG}{\neG}\geminateG{\geG}{\reG}}{({\teG}{\nG}{\naG}\geminateG{\geG}{\reG})} & \dotable{({\eG}{\xG}{\neG}\geminateG{\geG}{\reG})}{({\teG}{\xG}{\neG}\geminateG{\geG}{\reG})} \\ \hline 
  Imperfect   & {\yG}{\neG}{\gaG}{\gG}{\raG}{\lG}& \dotable{{\yG}\geminateG{\neG}{\gaG}\geminateG{\geG}{\raG}{\lG}}{({\yG}\geminateG{\neG}{\gaG}\geminateG{\geG}{\raG}{\lG})} & \dotable{{\yaG}{\neG}{\gaG}{\gG}{\raG}{\lG}}{{\yaG}\geminateG{\neG}{\gaG}\geminateG{\gG}{\raG}{\lG}}   & {\yaG}{\sG}{\neG}{\gaG}\geminateG{\gG}{\raG}{\lG} 
              & \dotable{({\yaG}{\sG}{\teG}{\naG}{\gaG}\geminateG{\gG}{\raG}{\lG})}{({\yG}{\teG}{\sG}{\teG}{\naG}\geminateG{\gG}{\raG}{\lG})}   & \dotable{{\yaG}{\nG}{\neG}\geminateG{\gG}{\raG}{\lG}}{({\yG}{\nG}{\neG}\geminateG{\geG}{\raG}{\lG})} & \dotable{({\eG}{\xG}{\neG}\geminateG{\geG}{\raG}{\lG})}{({\teG}{\xG}{\neG}\geminateG{\geG}{\raG}{\lG})} \\ \hline 
  Jussive     & {\yG}{\neG}{\gaG}{\geG}{\rG} &  \dotable{{\yG}\geminateG{\neG}{\gaG}{\geG}{\rG}}{({\yG}\geminateG{\neG}{\gaG}{\geG}{\rG})}& \dotable{{\yaG}{\nG}{\gaG}{\gG}{\rG}}{{\yaG}\geminateG{\neG}{\gaG}{\gG}{\rG}}     & {\yaG}{\sG}{\neG}{\gaG}{\gG}{\rG} 
              & \dotable{({\yaG}{\sG}{\teG}{\naG}{\gaG}{\gG}{\rG})}{({\yG}{\teG}{\sG}{\teG}{\naG}{\gG}{\rG})}  & \dotable{{\yaG}{\nG}{\naG}\geminateG{\gG}}{({\yG}{\nG}{\naG}{\geG}{\rG})}   & \dotable{({\eG}{\xG}{\neG}\geminateG{\geG}{\reG})}{({\teG}{\xG}{\neG}\geminateG{\geG}{\reG})} \\ \hline 
  Gerund      &   {\neG}{\gaG}{\gG}{\roG} &  \dotable{{\teG}{\neG}{\gaG}{\gG}{\roG}}{({\teG}{\neG}{\gaG}{\geG}{\roG})}& \dotable{{\eG}{\nG}{\gaG}{\gG}{\roG}}{{\eG}\geminateG{\neG}{\gaG}{\gG}{\roG}}     & {\eG}{\sG}{\neG}{\gaG}{\gG}{\roG} 
              & \dotable{({\eG}{\sG}{\teG}{\naG}{\gaG}{\gG}{\roG})}{({\teG}{\sG}{\teG}{\naG}{\gG}{\roG})}    & \dotable{{\eG}{\nG}{\naG}\geminateG{\goG}}{({\teG}{\nG}{\naG}\geminateG{\goG})}     & \dotable{({\eG}{\xG}{\neG}\geminateG{\geG}{\reG})}{({\teG}{\xG}{\neG}\geminateG{\geG}{\reG})} \\ \hline 
  NOA         &   {\neG}{\gaG}{\gaG}{\riG} &  \dotable{{\teG}{\neG}{\gaG}{\gaG}{\riG}}{({\teG}{\neG}{\gaG}{\gaG}{\riG})}& \dotable{{\eG}{\nG}{\gaG}{\gaG}{\riG}}{{\eG}\geminateG{\neG}{\gaG}{\gaG}{\riG}}     & {\eG}{\sG}{\neG}{\gaG}{\gaG}{\riG}  
              & \dotable{({\eG}{\sG}{\teG}{\naG}{\gaG}{\gaG}{\riG})}{({\teG}{\sG}{\teG}{\naG}{\gaG}{\riG})}    & \dotable{{\eG}{\nG}{\naG}{\gaG}{\riG}}{({\teG}{\nG}{\naG}{\gaG}{\riG})} & \dotable{({\eG}{\xG}{\neG}\geminateG{\geG}{\reG})}{({\teG}{\xG}{\neG}\geminateG{\geG}{\reG})} \\ \hline 
  Infinitive  & {\meG}{\neG}{\gaG}{\geG}{\rG} &  \dotable{{\meG}\geminateG{\neG}{\gaG}{\geG}{\rG}}{({\meG}\geminateG{\neG}{\gaG}{\geG}{\rG})}& \dotable{{\maG}{\nG}{\gaG}{\geG}{\rG}}{{\maG}\geminateG{\neG}{\gaG}{\geG}{\rG}}     & {\maG}{\sG}{\neG}{\gaG}{\geG}{\rG}  
              & \dotable{({\maG}{\sG}{\teG}{\naG}{\gaG}{\geG}{\rG})}{({\meG}{\teG}{\sG}{\teG}{\naG}{\geG}{\rG})}  & \dotable{{\eG}{\nG}{\naG}{\gaG}{\riG}}{({\teG}{\nG}{\naG}{\gaG}{\riG})} & \dotable{({\eG}{\xG}{\neG}\geminateG{\geG}{\reG})}{({\teG}{\xG}{\neG}\geminateG{\geG}{\reG})} \\ \hline 
  Instrumental&{\meG}{\neG}{\gaG}{\geG}{\riG}{\yaG}& \dotable{{\meG}\geminateG{\neG}{\gaG}{\geG}{\riG}{\yaG}}{({\meG}\geminateG{\neG}{\gaG}{\geG}{\riG}{\yaG})}& \dotable{{\maG}{\nG}{\gaG}{\geG}{\riG}{\yaG}}{{\maG}\geminateG{\neG}{\gaG}{\geG}{\riG}{\yaG}}     & ({\maG}{\sG}{\neG}{\gaG}{\geG}{\riG}{\yaG})  
              & \dotable{({\maG}{\sG}{\teG}{\naG}{\gaG}{\geG}{\riG}{\yaG})}{({\meG}{\teG}{\sG}{\teG}{\naG}{\geG}{\riG}{\yaG})} & \dotable{{\maG}{\nG}{\naG}{\geG}{\riG}{\yaG}}{({\meG}{\nG}{\naG}{\geG}{\riG}{\yaG})} & \dotable{({\eG}{\xG}{\neG}\geminateG{\geG}{\reG})}{({\teG}{\xG}{\neG}\geminateG{\geG}{\reG})} \\ \hline 

\end{tabular}\\



We may count now for our example, ``{\neG}{\geG}{\reG}'', 70 valid stems as seen in the two tables.  The number of realizable stems from the potential of 168 is again context dependent on the word.  A secondary classification for Amharic verbs could be created to describe the states realized by the stems in the above two tables.  To begin to consider the the possible number of derived stems we can make a quick estimation by considering that each stem above (or most as we shall see shortly) has also a negative form and inflections are possible for 11 personal pronouns in the subject and object.  This gives us 70x2x(12x11) or 18,480 derived forms for {\neG}{\geG}{\reG} (we used 12 for the count of the subject personal pronouns as they may stand alone without the object pronouns).  However in this estimation some unrealistic derivations are counted (such as 2$^{nd}$ person subject with 2${^nd}$ person object, the negative frequentative forms for {\neG}{\geG}{\reG} do not exist, and not all of our estimated derivations were orthographically unique.  In the rest of this article we will work on reducing these duplications as well as counting additional derivations that are specific to each part of speech.\\

The tables show the stems inflected for the 3$^{rd}$ person male except in the cases of the last three entries of the Noun of Agent, Infinitive, and the Instrumental (or Noun of Instrument) derivations which act like nouns and may recieve the possessive object markers.  Affixes for the 11 persons of Amharic are depicted in the following:



\section*{Personifications}
\begin{tabular}{|c|c|c|c|c|c|c|} \hline\hline
Person  &  Subject  & \dotable{Gerund}{Subject} &  \dotable{Present}{Perfect} 
        &  Object   & \dotable{Noun}{Possesive} & Contingent \\ \hline\hline
\begin{tabular}{c}$1^{st}$ Person \\ Singular  \\ {\IG}{\nEG}, {\neG}{\NG}   \end{tabular}   
        &  {\huG},{\kuG}    &  {\EG}       &({\yaG}$|${\AG}){\leG}{\huG}&   {\eG}{\NG}      &  {\EG},{\yEG},{\yeG}
        &  {\IG}-                  \\ \hline
\begin{tabular}{c}$2^{nd}$ Person \\ Masculine \\ {\eG}{\nG}{\teG}, {\neG}{\hG} \end{tabular}
        &  {\hG},{\kG}    &  {\hG}       &  {\haG}{\lG}       &   {\hG}        &  {\hG}   
        &  {\tG}-                  \\ \hline
\begin{tabular}{c}$2^{nd}$ Person \\ Feminine  \\ {\eG}{\nG}{\ciG}, {\neG}{\xG} \end{tabular}
        &  {\xG}       &  {\xG}       &  {\xaG}{\lG}       &   {\xG}        &  {\xG}          
        &  {\tG}$\cdots${\iG}         \\ \hline
\begin{tabular}{c}$2^{nd}$ Person \\ Demi-Polite \\ {\eG}{\nG}{\tuG}, {\neG}{\huG} \end{tabular}
        &  {\huG}       &  {\huG}       &  {\hWaG}{\lG}       &   {\huG}        &  {\huG}
        &  {\tG}$\cdots${\uG}         \\ \hline
\begin{tabular}{c}$2^{nd}$ Person \\ Polite    \\ {\IG}{\rG}{\sG}{\woG}, {\neG}{\woG}  \end{tabular}
        &  {\uG}       &  {\wG}       &  {\waG}{\lG}       &  {\eG}{\woG}({\tG})   &  {\woG}
        &  {\yG}$\cdots${\uG}         \\ \hline
\begin{tabular}{c}$3^{rd}$ Person \\ Masculine \\ {\IG}{\suG}, {\neG}{\wG}   \end{tabular}
        &  {\eG}       &  {\oG}       &  {\oG}{\waG}{\lG}     &  {\eG}({\wG},{\tG})  &  {\uG},{\wG}     
        &  {\yG}-                  \\ \hline
\begin{tabular}{c}$3^{rd}$ Person \\ Feminine  \\ {\IG}{\sWaG}, {\naG}{\tG}   \end{tabular}
        &  {\eG}{\cG}     &  {\AG}       &  {\AG}{\leG}{\cG}     &  {\AG}{\tG}       &  {\waG}        
        &  {\tG}-                  \\ \hline
\begin{tabular}{c}$3^{rd}$ Person \\ Polite    \\ {\IG}{\rG}{\saG}{\ceG}{\wG}, {\naG}{\ceG}{\wG}\end{tabular}
        &  {\uG}       &  {\wG}       &  {\waG}{\lG}       &  {\AG}{\ceG}{\wG}     &  {\AG}{\ceG}{\wG}    
        &  {\yG}$\cdots${\uG}         \\ \hline
\begin{tabular}{c}$1^{st}$ Person \\ Plural    \\ {\IG}{\NaG}, {\neG}{\nG}, {\neG}{\neG} \end{tabular}
        &  {\nG},{\neG}    &  {\nG}       &  {\naG}{\lG}       &  {\eG}{\nG}       &  {\AG}{\cG}{\nG}    
        &  {\IG}{\nG}-                \\ \hline
\begin{tabular}{c}$2^{nd}$ Person \\ Plural    \\ {\IG}{\naG}{\nG}{\teG}, {\naG}{\cG}{\huG} \end{tabular}
        &  {\AG}{\cG}{\huG}   &  {\AG}{\cG}{\huG}   &  {\AG}{\cG}{\hWaG}{\lG}   &  {\AG}{\cG}{\huG}     &  {\AG}{\cG}{\huG}    
        &  {\tG}$\cdots${\uG}         \\ \hline
\begin{tabular}{c}$3^{rd}$ Person \\ Plural    \\ {\IG}{\neG}{\rG}{\suG}, {\naG}{\ceG}{\wG} \end{tabular}
        &  {\uG}       &  {\wG}       &  {\waG}{\lG}       &  {\AG}{\ceG}{\wG}     &  {\AG}{\ceG}{\wG}    
        &  {\yG}$\cdots${\uG}         \\ \hline\hline
 Unique
        &  9/11     &  9/11     &  9/11       &  10/11      &  10/11    
        &  7/11                 \\ \hline\hline
\end{tabular}\\







The table shows us the written sequences for all personal inflections types found in Amharic.  The peronsal pronouns combine with stems, or with each other, following the normal rules of elisions in Amharic.  Amharic consonant-vowel and vowel-vowel elisions is beyond the scope of this paper.  The subject is an essential part of Amharic orthography and special considerations are needed.  The present perfect column is enough to demonstrate this point; except for the first person male and third person female the markers are the appendixation of ``-{\AG}{\lG}'' to the gerund subject pronouns. \\
%
% maybe talk about noun and contingent markers later.
%

We can see from the table also what suffixes are redundant.  In particular though {\IG}{\rG}{\saG}{\ceG}{\wG} and {\IG}{\neG}{\rG}{\suG} are different persons their personal markers are the same through out.  In some cases such as for the first person singular and plural and the second person male subject pronouns may differ.  This varies for region or time period in the case of the first person plural and in the other two cases depends on the syllable preceding the pronoun.  That {\kuG} or {\kG} is used after a 6$^{th}$ order syllable (as a consonant) or {\huG} and {\hG} are used elsewhere (where a \texttt{Cv} sequence must have occured).  Though two inflections exist for these persons only one can be used with a given stem.  Hence only 9 unique markers are potential to any given stem (or 8 when {\huG} occurs in the first person and is redundant with the demi-polite inflection -a lesser used form in modern urban Amharic).\\

Reductions occur elsewhere when subject and object are combined in unreal ways.  Taking the simple perfect as an example, the following table presents a comprehensive list of allowable combinations:



%\newcommand{\dotable}[2]{\begin{tabular}{c} #1 \\ #2 \end{tabular}}
%\clearpage
%\pagebreak
%\subsection*{Simple Perfect}
%\thispagestyle{landscape}
%\landscape
%\vspace*{-1.0in}
\hspace*{-1.65in}
{\textbf{\large Subject and Object Suffix Collisions in the Simple Perfect}} \\

\hspace*{-1.65in}
\noindent
%\begin{sidewaystable}
\begin{tabular}{*{2}{|@{\,}c@{\,}}||*{10}{@{\,}c@{\,}|}} \cline{3-12}
%  &    & \begin{tabular}{@{\,}c@{\,}}$1^{st}$ Per. \\ Singular  \\ {\IG}{\nEG}{\nG}    \end{tabular}
\multicolumn{2}{c|}{}   & \begin{tabular}{@{\,}c@{\,}}$1^{st}$ Per. \\ Singular  \\ {\IG}{\nEG}{\nG}    \end{tabular}
       & \begin{tabular}{@{\,}c@{\,}}$2^{nd}$ Per. \\ Masculine \\ {\eG}{\nG}{\teG}{\nG}  \end{tabular}
       & \begin{tabular}{@{\,}c@{\,}}$2^{nd}$ Per. \\ Feminine  \\ {\eG}{\nG}{\ciG}{\nG}  \end{tabular}
       & \begin{tabular}{@{\,}c@{\,}}$2^{nd}$ Per. \\ Polite    \\ {\IG}{\rG}{\sG}{\woG}{\nG}\end{tabular}
       & \begin{tabular}{@{\,}c@{\,}}$3^{rd}$ Per. \\ Masculine \\ {\IG}{\suG}{\nG}    \end{tabular}
       & \begin{tabular}{@{\,}c@{\,}}$3^{rd}$ Per. \\ Feminine  \\ {\IG}{\sWaG}{\nG}    \end{tabular}
       & \begin{tabular}{@{\,}c@{\,}}$3^{rd}$ Per. \\ Polite    \\ {\IG}{\rG}{\saG}{\ceG}{\wG}{\nG}\end{tabular}
       & \begin{tabular}{@{\,}c@{\,}}$1^{st}$ Per. \\ Plural    \\ {\IG}{\NaG}{\nG}    \end{tabular}
       & \begin{tabular}{@{\,}c@{\,}}$2^{nd}$ Per. \\ Plural    \\ {\IG}{\naG}{\nG}{\teG}{\nG}\end{tabular}
       & \begin{tabular}{@{\,}c@{\,}}$3^{rd}$ Per. \\ Plural    \\ {\IG}{\neG}{\rG}{\suG}{\nG}\end{tabular} \\ \cline{3-12} 

\multicolumn{2}{c|}{}
                & {\eG}{\NG}   & {\hG}     & {\xG}       & {\eG}{\woG}({\tG}) & {\eG}({\wG},{\tG}) 
                & {\AG}{\tG}   & {\AG}{\ceG}{\wG}   & {\eG}{\nG}     & {\AG}{\cG}{\huG}  & {\AG}{\ceG}{\wG}     \\ \hline \hline

\begin{tabular}{c}$1^{st}$ Per. \\ Singular \\ {\IG}{\nEG} \end{tabular}      
       & {\huG},{\kuG}  & \dotable{{\huG}{\NG}}{{\kuG}{\NG}}   & \dotable{{\huG}{\hG}}{{\kuG}{\hG}}     & \dotable{{\huG}{\xG}}{{\kuG}{\xG}}  & \dotable{{\huG}{\woG}{\tG}}{{\kuG}{\woG}{\tG}}   & \dotable{{\huG}{\tG}}{{\kuG}{\tG}}  
                & \dotable{{\hWaG}{\tG}}{{\kWaG}{\tG}}   & \dotable{{\hWaG}{\ceG}{\wG}}{{\kWaG}{\ceG}{\wG}} & NA                    & \dotable{{\hWaG}{\cG}{\huG}}{{\kWaG}{\cG}{\huG}}   & \dotable{{\hWaG}{\ceG}{\wG}}{{\kWaG}{\ceG}{\wG}}     \\ \hline

\begin{tabular}{c}$2^{nd}$ Per. \\ Masculine \\ {\eG}{\nG}{\teG} \end{tabular}
       & {\hG},{\kG}  & \dotable{\shadehalfcell{{\KeG}{\NG}}}{{\keG}{\NG}}  & \dotable{{\hG}}{{\kG}}         
                & NA                    & NA      & \dotable{\shadehalfcell{{\KeG}{\wG}}}{{\keG}{\wG}}  
                & \dotable{{\haG}{\tG}}{{\kaG}{\tG}}  & \dotable{{\haG}{\ceG}{\wG}}{{\kaG}{\ceG}{\wG}} 
                & \dotable{\shadehalfcell{{\KeG}{\nG}}}{{\keG}{\nG}}  & NA      & \dotable{{\haG}{\ceG}{\wG}}{{\kaG}{\ceG}{\wG}}     \\ \hline

\begin{tabular}{c}$2^{nd}$ Per. \\ Feminine \\ {\eG}{\nG}{\ciG}   \end{tabular}
       & {\xG}     & \shadecell{{\xG}{\NG}}  &  NA      & {\xG}       & NA      & \shadecell{{\xG}{\wG}}
                & {\xaG}{\tG}   & {\xaG}{\ceG}{\wG}   & \shadecell{{\xG}{\nG}}    & NA      & {\xaG}{\ceG}{\wG}     \\ \hline

\begin{tabular}{c}$2^{nd}$ Per. \\ Polite \\ {\IG}{\rG}{\sG}{\woG}   \end{tabular}
       & {\uG}     & {\uG}{\NG}   & NA       & NA       & {\uG}       & {\uG}{\tG}  
                & {\uG}{\waG}{\tG} & {\uG}{\waG}{\ceG}{\wG} & {\uG}{\nG}     & NA       & {\uG}{\waG}{\ceG}{\wG}  \\ \hline

\begin{tabular}{c}$3^{rd}$ Per. \\ Masculine \\ {\IG}{\suG}    \end{tabular}
       & {\eG}     & {\eG}{\NG}   & {\eG}{\hG}     & {\eG}{\xG}     & {\eG}{\woG}{\tG}   & {\eG}{\wG}   
                & {\AG}{\tG}   & {\AG}{\ceG}{\wG}   & {\eG}{\nG}     & {\AG}{\cG}{\huG}   & {\AG}{\ceG}{\wG}    \\ \hline

\begin{tabular}{c}$3^{rd}$ Per. \\ Feminine \\ {\IG}{\sWaG}  \end{tabular}
       & {\eG}{\cG}   & \shadecell{{\eG}{\cG}{\NG}}  & {\eG}{\cG}{\hG}   & {\eG}{\cG}{\xG}   & \shadecell{{\eG}{\cG}{\woG}{\tG}}& \shadecell{{\cG}{\wG}}
                & {\eG}{\caG}{\tG} & {\eG}{\caG}{\ceG}{\wG} & \shadecell{{\eG}{\cG}{\nG}}  & {\caG}{\cG}{\huG}  & {\eG}{\caG}{\ceG}{\wG}   \\ \hline

\begin{tabular}{c}$3^{rd}$ Per. \\ Polite \\ {\IG}{\rG}{\saG}{\ceG}{\wG} \end{tabular}
       & {\uG}     & {\uG}{\NG}   & {\uG}{\hG}     & {\uG}{\xG}     & {\uG}{\woG}{\tG}   & {\uG}{\tG}  
                & {\uG}{\waG}{\tG} & {\uG}{\waG}{\ceG}{\wG} & {\uG}{\nG}     & {\uG}{\waG}{\cG}{\huG} & {\uG}{\waG}{\ceG}{\wG} \\ \hline

\begin{tabular}{c}$1^{st}$ Per. \\ Plural \\ {\IG}{\NaG}       \end{tabular}
       & {\eG}{\nG}   & NA     & {\eG}{\nG}{\hG}   & {\eG}{\nG}{\xG}   & {\eG}{\nG}{\woG}{\tG} & {\eG}{\neG}{\wG}
                & {\eG}{\naG}{\tG} & {\eG}{\naG}{\ceG}{\wG} & {\eG}{\nG}     & {\eG}{\naG}{\cG}{\huG} & {\eG}{\naG}{\ceG}{\wG} \\ \hline

\begin{tabular}{c}$2^{nd}$ Per. \\ Plural \\ {\IG}{\naG}{\nG}{\teG}   \end{tabular}
       & {\AG}{\cG}{\huG} &{\AG}{\cG}{\huG}{\NG}& NA       & NA       & NA       & {\AG}{\cG}{\huG}{\tG}   
                &{\AG}{\cG}{\hWaG}{\tG}&{\AG}{\cG}{\hWaG}{\ceG}{\wG}& {\AG}{\cG}{\huG}{\nG} & {\AG}{\cG}{\huG}   & {\AG}{\cG}{\hWaG}{\ceG}{\wG}\\ \hline

\begin{tabular}{c}$3^{rd}$ Per. \\ Plural  \\ {\IG}{\neG}{\rG}{\suG}  \end{tabular}
       & {\uG}     & {\uG}{\NG}   & {\uG}{\hG}     & {\uG}{\xG}     & {\uG}{\woG}{\tG}   & {\uG}{\tG}  
                & {\uG}{\waG}{\tG} & {\uG}{\waG}{\ceG}{\wG} & {\uG}{\nG}     & {\uG}{\waG}{\cG}{\huG} & {\uG}{\waG}{\ceG}{\wG}  \\ \hline
Unique &  9/11  &  8/11  &  6/11    &  6/11    &  6/11    & 8/11
                &  8/11  &  8/11    &  6/11    &  5/11    & 8/11      \\ \hline
\end{tabular}  \\



%\noindent
%\begin{tabbing}
%Note: \=the cluster {\uG}+{\AG} = {\uG}{\waG} is recognized as \\ %%{\wWaG}. \\
%      \>In example {\quG}+{\AG} = {\qWaG}, {\huG}+{\AG}={\hWaG}
%\end{tabbing}

%\caption{Suffixation Unions}
%\global\def\rotfloatpage{R}
%\end{sidewaystable}

%% Repeat For Gerund





  What we see then is 78 unique sequences from a potential of 121.  Two columns duplicate however (the 3$^{rd}$ person polite and plural) so we reduce again to  70 unique sequences.  \textit{Note: A demi-polite row would duplicate the 1$^{st}$ person singular, we should have a demi-polite column as an object, add this and recalculate}.  We may now caculate exactly for the themes of the simple perfect a contribution of 10x1.5x70 or 1,050 derived words to our original quick estimation (1.5 accounts for the non-existance of the frequentative forms in the negative).\\

To go from an estimation to a precise figure for all derived forms in the simple perfect we need to consider the other available modifiers.  Out next table should comprehensively indicate other modifiers possible in the simple perfect:\\
 
\subsection*{Derived Verbs in the Simple Perfect}
\begin{tabular}{|r|c|c|l|} \hline\hline
  Prefix            & \dotable{Required}{Midfix} & Verb Stem & Allowable Suffixes  \\ \hline 
{\yeG}                  & +None+ & $<$verb$>$ & \lbbet + {\nG} + \continuants  \\
(None,{\leG},{\beG},{\keG},{\sG}{\lG}{\spaceG},{\IG}{\nG}{\dG}{\spaceG}) 
                    & +None+ & $<$verb$>$ & \lbbet + \continuants       \\ \hline
 None               & +{\eG}{\lG}+ & $<$verb$>$ & \lbbet + (({\mG}+{\sG}),(({\mG},{\sG}) + {\naG})) \\
 {\yeG}                 & +{\eG}{\lG}+ & $<$verb$>$ & \lbbet + {\nG} + \continuants  \\
({\leG},{\beG},{\keG},{\sG}{\lG},{\IG}{\nG}{\dG},{\IG}{\sG}{\kG})
                    & +{\eG}{\lG}+ & $<$verb$>$ & \lbbet + \continuants       \\ \hline\hline
\end{tabular} \\


It is assumed in the $<$verb$>$ stem that it is alread inflected for the subject.  The first row with {\yeG} gives us 2,954 possible derivations the second row offers another 5,908 outcomes.  {\IG}{\nG}{\dG}{\spaceG} and {\sG}{\lG}{\spaceG} are shown here as detached from the word and are \textit{not} counted in the figure just given.  They may attatch under informal writing practices or when the verb stem starts with a vowel as we will see in the negative.  The first row of the negative conjugations offers an additional 1,477 outcomes.  The second row another 2,954 and finally 8,862 from out last row.  This sum comes to 22,155 derived forms for a single them of the simple perfect.  The frequentative form contributes only in the positve for another 8,862 realized words bringing the total now to 31,017 for a single them.  To account for the other realized themes of {\neG}{\geG}{\reG} we can multiply this figure again by 5 to find 155,085 derived possibilities (\textit{quote again the extremum potential}).

\subsection*{Derived Verbs in the Imperfect}
\begin{tabular}{|r|c|c|l|} \hline\hline
  Prefix  & \dotable{Required}{Midfix} & Verb Stem & Allowable Suffixes  \\ \hline 
  None    & +None+ & $<${\yeG}{\eG}{\huG}{\nG}{\spaceG}{\gG}{\zEG}$>$  & \lbbet + \continuants  \\  \hline  
          &        &                       & negative -see contingent   \\ \hline\hline
\end{tabular} \\


\hspace*{-1.0in}
\subsection*{Derived Verbs in the Relative}
\hspace*{-1.0in}
\begin{tabular}{|r|c|c|l|} \hline\hline
  Prefix                 & \dotable{Required}{Midfix} & Verb Stem & Allowable Suffixes  \\ \hline
{\yeG}                               &+[{\mG}] + {\eG}+& $<$Cont.$>$ & (\lbbet,{\wG}/{\waG}) + {\nG} + \continuants  \\
({\leG},{\beG},{\keG},{\teG},{\sG}{\lG},{\IG}{\nG}{\dG},{\IG}{\sG}{\kG}) &+[{\mG}] + {\eG}+& $<$Cont.$>$ & (\lbbet,{\wG}/{\waG}) + \continuants  \\
({\yeG},{\keG},{\leG})                       &+[{\IG}{\nG}{\dG} + {\mG}] + {\eG}+& $<$Cont.$>$ & [\lbbet,{\wG}/{\waG}] + \continuantsx  \\ \hline\hline
\end{tabular} \\


\subsection*{Derived Verbs in the Past Participle (Gerund)}
\begin{tabular}{|r|c|c|l|} \hline\hline
  Prefix            & \dotable{Required}{Midfix} & Verb Stem & Allowable Suffixes  \\ \hline 
  None   &  +None+  & $<${\boG}{\zG}$>$       & \lbbet + \continuants \\ \hline
  [{\sG}]   &  +{\eG}+    & $<$Contingent$>$ & \lbbet + \continuants \\ \hline\hline
\end{tabular} \\


\subsection*{Derived Verbs in the Present Perfect}
\begin{tabular}{|r|c|c|l|} \hline\hline
  Prefix & \dotable{Required}{Midfix}    & Verb Stem & Allowable Suffixes  \\ \hline  
  None   & +None+ & $<${\boG}{\zG}$>$           & \lbbet + [{\eG}{\lG}] + \continuants  \\ \hline 
  None   & +None+ & $<${\yeG}{\hheG}{\laG}{\fiG}{\spaceG}{\gG}{\zEG}$>$ & see negative simple perfect     \\ \hline\hline
\end{tabular} \\


\subsection*{Derived Verbs in the Jussive}
\begin{tabular}{|r|c|c|l|} \hline\hline
  Prefix  & \dotable{Required}{Midfix} & Verb Stem & Allowable Suffixes  \\ \hline 
  None    &  +None+   & $<verb>$       & \lbbet + \continuants           \\ \hline 
  None    &  +{\eG}+     & $<verb>$       & \lbbet + \continuants           \\ \hline\hline
\end{tabular} \\


\subsection*{Derived Verbs in the Contingent}
\begin{tabular}{|l|l|l|l|} \hline\hline
  Affix  & \dotable{Allowable}{Prefixes}  & \dotable{Allowable}{Suffixes} & Meaning \\ \hline 
  None                &        &                               &  \\
({\lG},{\bG},{\IG}{\nG}{\dG},{\IG}{\sG}{\kG}) &        & \lbbet + ({\mG},{\sG}) + {\naG}   & \\
 {\sG}                   &        & \lbbet + ({\mG},{\sG})        & \\
 {\yeG}{\mG}-                &        &                         & see relative   \\
 {\eG}-         & ({\lG},{\bG},{\IG}{\nG}{\dG})  & \lbbet + ({\mG},{\sG}) + {\naG}   & negative \\
             & {\sG}              & (([({\lG},{\bG})+OP,{\beG}{\tG}] + ({\mG},{\sG},{\naG})), ({\mG},{\sG})) & negative \\
             & {\IG}{\sG}{\kG} + [{\mG}]   & (([({\lG},{\bG})+OP,{\beG}{\tG}] + ({\mG},{\sG},{\naG})), {\naG}) & \\ \hline\hline
\end{tabular} \\

\subsection*{Derived Verbs in the Infinitive}
\begin{tabular}{|r|c|c|l|} \hline\hline
  Prefix                   & \dotable{Required}{Midfix} & Verb Stem & Allowable Suffixes  \\ \hline  
 ({\yeG},None)                 & +None+ & $<${\eG}{\rG}{\IIG}{\sG}{\tG}$>$ & POS + {\nG} + \continuantsy       \\
 ({\leG},{\keG},{\sG}{\lG}{\spaceG},{\IG}{\nG}{\dG}{\spaceG})   & +None+ & $<${\eG}{\rG}{\IIG}{\sG}{\tG}$>$ & POS + \continuants          \\
 {\IG}{\sG}{\kG}{\spaceG}                  & +None+ & $<${\eG}{\rG}{\IIG}{\sG}{\tG}$>$ & POS + \continuants          \\  \hline
 ({\yeG},None)                 & +{\eG}{\leG}+ & $<${\eG}{\rG}{\IIG}{\sG}{\tG}$>$ & POS\tinyn + {\nG}+ \continuants      \\
 ({\beG},{\keG},{\leG},{\sG}{\lG}{\spaceG},{\IG}{\nG}{\dG}{\spaceG})& +{\eG}{\leG}+ & $<${\eG}{\rG}{\IIG}{\sG}{\tG}$>$ & POS + \continuants          \\  \hline\hline
\end{tabular} \\

\noi
\hspace*{-1.0in}{\large\bf Derived Nouns of Agent}\\
\noi
\hspace*{-1.0in}
\begin{tabular}{|r|c|c|l|} \hline\hline 
  Prefix                    & \dotable{Required}{Midfix} & Noun Stem & Allowable Suffixes  \\ \hline 

  (None,{\yeG})                 & None     & NOA & ({\neG}{\tG},PLU) + (({\iG}{\tG}+[DEF]\tinyit),POS) + {\nG} + \continuants \\
  ({\leG},{\beG},{\keG},{\IG}{\nG}{\dG})         & None     & NOA & ({\neG}{\tG},PLU) + (({\iG}{\tG}+[DEF]\tinyit),POS) + \continuants \\ 
  {\IG}{\sG}{\keG}{\spaceG}                  & None     & NOA & PLU + (({\iG}{\tG}+[DEF]\tinyit),POS) + \continuants \\ \hline

  {\yeG}                        & +{\IG}{\nG}{\dG}{\spaceG}& NOA & ({\neG}{\tG},PLU) + [({\iG}{\tG}+[DEF]\tinyit),POS]\tinyInd + {\nG} + \continuants \\
  ({\leG},{\beG},{\keG})                & +{\IG}{\nG}{\dG}{\spaceG}& NOA & ({\neG}{\tG},PLU) + [({\iG}{\tG}+[DEF]\tinyit),POS]\tinyInd + \continuants \\ \hline

  {\yeG}                        & +{\IG}{\yeG}+   & NOA & PLU + [({\iG}{\tG}+[DEF]\tinyit),POS]\tinyIye + {\nG} + \continuants \\ 
  ({\leG},{\beG},{\keG},{\sG}{\lG},{\IG}{\nG}{\dG})    & +{\IG}{\yeG}+   & NOA & PLU + [({\iG}{\tG}+[DEF]\tinyit),POS]\tinyIye + \continuants \\ \hline

  (None,{\yeG})                 & +{\IG}{\neG}+   & NOA & PLU + [({\iG}{\tG}+[DEF]\tinyit),POS]\tinyIne  + {\nG} + \continuants \\
  ({\leG},{\beG},{\keG},{\sG}{\lG},{\IG}{\sG}{\kG},{\IG}{\nG}{\dG})
                            & +{\IG}{\neG}+   & NOA & PLU + [({\iG}{\tG}+[DEF]\tinyit),POS]\tinyIne  + \continuants \\ \hline

  ({\beG},{\keG},{\yeG})                & +{\eG}{\leG}{\spaceG}  & NOA & PLU + [({\iG}{\tG}+[DEF]\tinyit),POS]\tinyale + \continuants \\ \hline\hline
\end{tabular}\\
\noi
[DEF] can not follow {\neG}{\tG}, except in some agentive nouns the definite is required
in the absence of any prefixes. ({\neG}{\tG},PLU), PLU and {\iG}{\tG} can not combine.\\


\vspace{0.20in}
\noi
\hspace*{-1.0in}{\large\bf Derived Material Nouns}\\
\noi
\hspace*{-1.0in}
\begin{tabular}{|r|c|c|l|} \hline\hline 
  Prefix                  & \dotable{Required}{Midfix} & Noun Stem & Allowable Suffixes  \\ \hline 

  (None,{\yeG})               & None     & Material Noun, NOI & ({\neG}{\tG},PLU) + (DEF,POS) + {\nG} + \continuants \\
({\leG},{\beG},{\keG},{\sG}{\lG}{\spaceG},{\IG}{\nG}{\dG}{\spaceG})& None     & Material Noun, NOI & ({\neG}{\tG},PLU) + (DEF,POS) + \continuants \\   
  {\IG}{\sG}{\kG}                  & None     & Material Noun, NOI & ({\mG},{\sG}) \\ \hline

  {\yeG}                      & +{\IG}{\nG}{\dG}{\spaceG}& Material Noun, NOI & ({\neG}{\tG},PLU) + (DEF,POS) + {\nG} + \continuants \\
  ({\leG},{\beG},{\keG})              & +{\IG}{\nG}{\dG}{\spaceG}& Material Noun, NOI & ({\neG}{\tG},PLU) + (DEF,POS) + \continuants \\ \hline

  {\yeG}                      & +{\IG}{\yeG}+   & Material Noun, NOI & [PLU]\tinyIye + [DEF,POS]\tinyIye + {\nG} + \continuants \\
  ({\leG},{\beG},{\keG},{\sG}{\lG},{\IG}{\nG}{\dG})  & +{\IG}{\yeG}+   & Material Noun, NOI & [PLU]\tinyIye + [DEF,POS]\tinyIye + \continuants \\ \hline

  (None,{\yeG})               & +{\IG}{\neG}+   & Material Noun, NOI & {\nG} + \continuants \\
  ({\leG},{\beG},{\sG}{\lG},{\IG}{\nG}{\dG})     & +{\IG}{\neG}+   & Material Noun, NOI &      \continuants \\
  ({\IG}{\sG}{\kG},{\keG})             & +{\IG}{\neG}+   & Material Noun, NOI & PLU + [DEF,POS]\tinyIne + \continuants \\
  ({\keG},{\yeG}) +{\eG}{\leG}+          & +{\IG}{\neG}+   & Material Noun, NOI & \continuants \\ \hline

  ({\keG},{\yeG})                 & +{\eG}{\leG}{\spaceG}  & Material Noun, NOI & (DEF,POS) + \continuants \\ \hline\hline
\end{tabular}
\noi
For the NOI a prefix is required when {\neG}{\tG} is used without the DEF.\\
When no prefix is present on the NOI, DEF is required after {\neG}{\tG}.
% The use of {\neG} here also becomes personifying so PLU + (DEF,POS) does not follow
% except with ({\IG}{\sG}{\kG},{\kG}).

\end{document}








